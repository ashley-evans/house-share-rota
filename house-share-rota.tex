\documentclass[DIV=13, landscape]{scrartcl}
\usepackage{graphicx, cals}
\usepackage{changepage}
\usepackage{pdflscape}
\usepackage[british]{babel}
\usepackage[useregional, calc, datesep=/]{datetime2}
\usepackage{pgffor}
\usepackage{arrayjob}
\usepackage{intcalc}

% Remove Page Numbers.
\pagenumbering{gobble}

% Function: Add Days
\newcount\daycount
\newcommand{\AddDays}[2]{%
	\DTMsavedate{CurrentDate}{#1}%
	\DTMsaveddateoffsettojulianday{CurrentDate}{#2}{\daycount}%
	\DTMsavejulianday{CurrentDate}{\number\daycount}%
	\DTMusedate{CurrentDate}%
}%

% Update date here to change initial date of rota. ISO Format.
% The initial date is the start date of the first rota produced.
\newcommand{\InitialDate}{2020-07-20}

\DTMsavedate{InitialDate}{\InitialDate}

% Initial array order for initial date.
\newarray\initials
\readarray{initials}{A&F&So&Si&G&J&V}

% Update date here to change start date of rota. ISO Format.
% The start date is for the rota being currently produced.
\newcommand{\StartDate}{2020-07-20}

\DTMsavedate{StartDate}{\StartDate}

% Calculate difference between initial and start to offset array order.
\newcount\offsetdaycount
\DTMsaveddatediff{StartDate}{InitialDate}{\offsetdaycount}

% Update here to change number of days to display past initial date.
\newcommand{\DaysToDisplay}{13}
\newcommand{\TotalRows}{\intcalcAdd{\DaysToDisplay}{1}}

% Calculate how many full pages to print (14 rows per page)
\newcommand{\FullPagesToPrint}{\intcalcDiv{\TotalRows}{14}}

% Calculate how many partial pages to print (Less than 14 rows)
\newcommand{\PartialPagesToPrint}{\intcalcMin{\intcalcMod{\TotalRows}{14}}{1}}

% Calculate total number of pages
\newcommand{\TotalPagesToPrint}{\intcalcAdd{\FullPagesToPrint}{\PartialPagesToPrint}}

\begin{document}

\makeatletter

% Define cell border widths
\def\cals@cs@width{1pt}
\def\cals@rs@width{1pt}
		
% Define frame border widths
\def\cals@framers@width{1pt}
\def\cals@framecs@width{1pt}
		
% Define cell background colour to default empty
\def\cals@bgcolor{}

\foreach \i in {0,...,\intcalcSub{\TotalPagesToPrint}{1}}%
{
	% Calculate lower bound of current rota page.
	\newcommand{\PageLowerBound}{\intcalcMul{\i}{14}}
	
	% Calculate upper bound of current rota page.
	\newcommand{\PageUpperBound}{\intcalcMin{\intcalcAdd{\PageLowerBound}{13}}{\DaysToDisplay}}
	
	\textbf{Rota for: \AddDays{\StartDate}{\PageLowerBound} - \AddDays{\StartDate}{\PageUpperBound}}

	\begin{calstable}[c]
		
		% Handle column sizing.	
		\colwidths{
			{\dimexpr(\columnwidth)/78*30\relax} % Date Column Size
			{\dimexpr(\columnwidth)/78*8\relax} % Task 1 Column Size
			{\dimexpr(\columnwidth)/78*8\relax} % Task 2 Column Size
			{\dimexpr(\columnwidth)/78*8\relax} % Task 3 Column Size
			{\dimexpr(\columnwidth)/78*8\relax} % Task 4 Column Size
			{\dimexpr(\columnwidth)/78*8\relax} % Task 5 Column Size
			{\dimexpr(\columnwidth)/78*8\relax} % Task 6 Column Size
		}
		
		% Header rows
		\thead{
			% Header row 1 
			\brow
				\nullcell{tlr}
				
				\nullcell{tlb}
				\nullcell{tb}
				\nullcell{tb}
				\nullcell{tb}
				\nullcell{tb}
				\nullcell{trb} \alignC \spancontent{\textbf{Tasks}}
			\erow
			
			% Header row 2
			\brow
				\nullcell{tlbr} \alignC \spancontent{\vfill \textbf{Date}}
				
				\cell{\vfil \rotatebox[origin=c]{90}{\textbf{Change Bins}}} \alignC
				\cell{\vfil \rotatebox[origin=c]{90}{\textbf{Sweep Floor}}} \alignC
				\cell{\vfil \rotatebox[origin=c]{90}{\textbf{Wipe Surfaces}}} \alignC
				\cell{\vfil \rotatebox[origin=c]{90}{\textbf{Wipe appliances}}} \alignC
				\cell{\vfil \rotatebox[origin=c]{90}{\textbf{Clean Microwave}}} \alignC
				\cell{\vfil \rotatebox[origin=c]{90}{\textbf{Handle Dishwasher}}} \alignC
			\erow
		}
		
		% Body Rows
		\foreach \j in {\PageLowerBound,...,\PageUpperBound}
		{
			\newcommand{\offsetJ}{\intcalcAdd{\j}{\number\offsetdaycount}}	
		
			\brow
				\cell{\textbf{\AddDays{\InitialDate}{\offsetJ}}}
	
				\cell{\initials(\intcalcAdd{\intcalcMod{0 - \offsetJ}{7}}{1})}
				\cell{\initials(\intcalcAdd{\intcalcMod{1 - \offsetJ}{7}}{1})}
				\cell{\initials(\intcalcAdd{\intcalcMod{2 - \offsetJ}{7}}{1})}
				\cell{\initials(\intcalcAdd{\intcalcMod{3 - \offsetJ}{7}}{1})}
				\cell{\initials(\intcalcAdd{\intcalcMod{4 - \offsetJ}{7}}{1})}
				\cell{\initials(\intcalcAdd{\intcalcMod{5 - \offsetJ}{7}}{1})}
			\erow
		}%
	\end{calstable}
}

\end{document}
