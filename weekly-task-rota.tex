\documentclass[DIV=13, landscape]{scrartcl}
\usepackage{graphicx, cals}
\usepackage{xcolor}
\usepackage[british]{babel}
\usepackage[useregional, calc]{datetime2}
\usepackage{pgffor}
\usepackage{arrayjob}
\usepackage{intcalc}
\usepackage{pifont}
\usepackage{ifthen}

% Cross-Mark and Tick-Mark command.
\newcommand{\xmark}{\ding{55}}
\newcommand{\tmark}{\ding{51}}

% Remove Page Numbers.
\pagenumbering{gobble}

% Function: Add Days
\newcount\daycount
\newcommand{\AddDays}[2]{%
	\DTMsavedate{CurrentDate}{#1}%
	\DTMsaveddateoffsettojulianday{CurrentDate}{#2}{\daycount}%
	\DTMsavejulianday{CurrentDate}{\number\daycount}%
	\DTMusedate{CurrentDate}%
}%

% Function: Add Weeks
\newcommand{\AddWeeks}[2]{%
	\AddDays{#1}{\intcalcMul{#2}{7}}%
}%

% Update date here to change initial date of rota. ISO Format.
% The initial date is the start date of the first rota produced (Sunday).
\newcommand{\InitialDate}{2020-07-26}

% Initial array order for initial date.
\newarray\initials
\readarray{initials}{V&A&F&Si&G&So&J}

% Update flag here to change whether initial date is an all bin week, 1 = true/0 = false.
\newcommand{\isInitialWeekAllBins}{1}

\DTMsavedate{InitialDate}{\InitialDate}

% Update date here to change start date of rota. ISO Format.
% The start date is for the rota being currently produced.
\newcommand{\StartDate}{2020-07-26}

\DTMsavedate{StartDate}{\StartDate}

% Calculate week difference between initial and start to offset row output.
\newcount\offsetdaycount
\DTMsaveddatediff{StartDate}{InitialDate}{\offsetdaycount}
\newcommand{\offsetweekcount}{\intcalcDiv{\number\offsetdaycount}{7}}

% Update here to change number of weeks to display past initial date.
\newcommand{\WeeksToDisplay}{13}

\begin{document}

\makeatletter

% Define Background swaps
\newcommand{\bglightgray}{
	\ifx\cals@bgcolor\empty
		\def\cals@bgcolor{gray!15}
	\else 
		\def\cals@bgcolor{} 
	\fi
}

\newcommand{\bgdarkgray}{
	\ifx\cals@bgcolor\empty
		\def\cals@bgcolor{gray!30}
	\else 
		\def\cals@bgcolor{} 
	\fi
}

% Define cell border widths
\def\cals@cs@width{1pt}
\def\cals@rs@width{1pt}
		
% Define frame border widths
\def\cals@framers@width{1pt}
\def\cals@framecs@width{1pt}
		
% Define cell background colour to default empty
\def\cals@bgcolor{}

\textbf{Weekly rota for: \DTMusedate{StartDate} - \AddWeeks{\StartDate}{\WeeksToDisplay}}

\begin{calstable}[c]
	
	% Handle column sizing.	
	\colwidths{
		{\dimexpr(\columnwidth)/78*24\relax} % Date Column Size
		{\dimexpr(\columnwidth)/78*15\relax} % Task 1 Column Size
		{\dimexpr(\columnwidth)/78*15\relax} % Task 2 Column Size
		{\dimexpr(\columnwidth)/78*24\relax} % Assignment Column Size
	}
	
	% Header rows
	\thead{
		% Header row 1 
		\brow
			\bglightgray \nullcell{tlr}
			
			\nullcell{tlb}
			\nullcell{trb} \alignC \spancontent{\textbf{Tasks}}
			\nullcell{tlr} \bglightgray
		\erow
		
		% Header row 2
		\brow
			\bglightgray \nullcell{tlbr} \alignC \spancontent{\vfill \textbf{Date}} \bglightgray
			
			\bgdarkgray \cell{\textbf{Black Bins}} \alignC
			\cell{\textbf{Recycling}} \alignC \bgdarkgray
			\bglightgray \nullcell{lbr} \alignC \spancontent{\vfill \textbf{Assignment}} \bglightgray
		\erow
	}
	
	\foreach \i in {0,...,\WeeksToDisplay}
	{
		\newcommand{\offsetI}{\intcalcAdd{\i}{\offsetweekcount}}	
		
		% Calculate if week is all bin week, 1 = true/0 = false.	
		\newcommand{\isAllBinWeek}{\intcalcSub{\isInitialWeekAllBins}{\intcalcMod{\offsetI}{2}}}		
		
		\brow
			\cell{\textbf{\AddWeeks{\InitialDate}{\offsetI}}}
	
			\cell{\ifthenelse{\isAllBinWeek = 1}{\tmark}{\xmark}}
			\cell{\tmark}
			\cell{\textbf{\initials(\intcalcAdd{\intcalcMod{\intcalcAdd{0}{\intcalcMul{\offsetI}{2}}}{7}}{1}) \& \initials(\intcalcAdd{\intcalcMod{\intcalcAdd{1}{\intcalcMul{\offsetI}{2}}}{7}}{1})}}
		\erow
	}
\end{calstable}

\end{document}
